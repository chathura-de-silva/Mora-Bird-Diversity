\chapter{Comparison}
\label{cp:Comparative}
\section{Species}
In the investigation conducted roughly two decades ago, 83 bird species were documented[1]. In the present study, 16 additional bird species, including two endemic species, were identified that were not recorded in the previous study.

\begin{importantbox}
\subsection{Species recorded for the first time}
\begin{enumerate}
    \item Black-Winged Stilt - \textit{Himantopus himantopus}
    \item Brown-breasted Flycatcher - \textit{Muscicapa muttui}
    \item Common Moorhen - \textit{Gallinula chloropus}
    \item Eastern/Western Yellow Wagtail - \textit{Motacilla tschutschensis/flava}
    \item Gray-bellied Cuckoo - \textit{Cacomantis passerinus}
    \item Great Cormorant - \textit{Phalacrocorax carbo}
    \item Indian Golden Oriole - \textit{Oriolus kundoo}
    \item Indian Peafowl - \textit{Pavo cristatus}
    \item Jerdon's Leafbird - \textit{Chloropsis jerdoni}
    \item Little Tern - \textit{Sternula albifrons}
    \item Oriental Darter - \textit{Anhinga melanogaster}
    \item Oriental Honey-buzzard - \textit{Pernis ptilorhynchus}
    \item Rock Pigeon (Feral) - \textit{Columba livia}
    \item Sri Lanka Green-Pigeon -\textit{Treron pompadora}
    \item Sri Lanka Swallow - \textit{Cecropis hyperythra}
    \item Watercock- \textit{Gallicrex cinerea}
\end{enumerate}
\end{importantbox}

Conversely, there are 11 species documented in the earlier study that were not recorded in the current research. This includes a single endemic species, Sri Lanka Grey Hornbill(\textit{Ocyceros gingalensis}).
