\chapter{Overview}
\label{cp:overview}
\section{Data Collection Methodology}
Data collection for this study took place from 2019 to December 2023 by one author and from September 2022 to mid-January 2024 by the other. The process involved random observations carried out throughout these periods, with some birds identified based on their distinct calls.

To create the final comprehensive checklist of bird species, the data collected by both authors was merged.

\begin{importantbox}
\subsection{Limitations}
 Random observations inherently possess limitations, compounded by the potential for overlooking certain university regions due to challenging accessibility. Furthermore, observations across various areas may lack uniform frequency, introducing variability. Additionally, two authors conducted overlapping yet distinct observations during disparate time periods.
\end{importantbox}

\section{Climate}
University is located in the Low country Wet Zone. The area experiences an annual precipitation, primarily brought by the southwest monsoon from May to September. Additionally, inter-monsoon rains occur in March-April and October-November. Although January tends to be drier, there is no distinctly defined dry season. The average temperature hovers around 27 degrees Celsius, accompanied by a relative humidity of approximately 90\%.

\section{Eco-systems}
The observation reveals that various bird species exhibit a diverse range of habitat preferences, encompassing both entirely natural ecosystems and human-made ecosystems situated within and in close proximity to the university.

The neighboring regions of Bolgoda Lake and the Kaju Kele area can be categorized as natural ecosystems, whereas spaces like gardens and the university playground fall under the classification of human-made ecosystems. Notably, even in entirely artificial environments, the study records the presence of certain bird species while some of the species are strictly restricted to certain areas of the university.
