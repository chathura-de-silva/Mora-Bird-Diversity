\chapter{Conclusion}

The University of Moratuwa, situated in a densely urbanized locale, surprisingly sustains a notably diverse avian population. The coexistence of a thriving natural environment within the university grounds and the encompassing Bolgoda Lake Eco System significantly contributes to this ecological richness.
\\\\
The conservation significance of these natural ecosystems is underscored by the documented presence of endemic bird species exclusive to Sri Lanka in addition to the fact that from 2003 to 2024, the recorded bird species adds up to a total of 93.
\\\\
Upon comparison with a previous study, despite potential limitations in the data collection methodology that may have led to overlooking certain rare species, the absence of some previously common species in the current study warrants further in depth observation and consideration. Notably, the emergence of the Rock Pigeon as a prevalent species within the university premises, as indicated in this documentation, also deserves attention.