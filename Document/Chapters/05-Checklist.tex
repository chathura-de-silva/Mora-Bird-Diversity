\chapter{Checklist}
\label{cp:checklist}
Following is the comprehensive checklist of bird species recorded in the university premises of University of Moratuwa according to the collected data of sightings from 2019 to 2024.

\begin{itemize}%
\item%
 Accipitridae%
\begin{enumerate}%
\item%
\begin{description}%
\item[]%
\textit{Haliastur indus (LC)}%
\item[]%
\textbf{Brahminy Kite/Red Backed Sea Eagle}%
\end{description}%
\begin{description}%
\item[H: ]%
Very common breeding resident in lowlands. Regular but rare visitor to the hills. Can be observed easily near water mainly in coastal areas and large inland wetlands and also along rivers.%
\item[D: ]%
Fish,Snakes, lizards, frogs, and small mammals. They soar high in the air and swoop down to catch their prey.%
\item[R: ]%
Most commonly can  be observed during flight in anywhere in the university sky. Perched specimen can be spotted near the university ground premises and the bolgoda lakeside trees.%
\end{description}%
\item%
\begin{description}%
\item[]%
\textit{Accipiter badius (LC)}%
\item[]%
\textbf{Shikra/Little Banded Goshawk}%
\end{description}%
\begin{description}%
\item[H: ]%
A common breeding resident throughout the entire island. Can be observed in both open country and in dense forest. Specially can be observed in close proximity with residential areas too.%
\item[D: ]%
 Lizards, birds, snakes, and insects. They hunt from perches or by flying low over the ground.%
\item[R: ]%
Trees around the Sumanadasa building and the Lagan.%
\end{description}%
\item%
\begin{description}%
\item[]%
\textit{Pernis ptilorhynchus (VU)}%
\item[]%
\textbf{Oriental Honey Bazzard}%
\end{description}%
\begin{description}%
\item[H: ]%
Fairly rare breeding resident and the population much increased by the winter migrants. Occurs throughout the Island. Mostly observed in well{-}wooded areas.%
\item[D: ]%
Primarily larvae, pupae, and honeycombs of social wasps and bees, occasionally supplemented with cicadas, small birds, reptiles, and frogs. They dig up wasp nests with their strong talons and long beak, adapted for extracting prey from combs.%
\item[R: ]%
Edges of the university ground%
\end{description}%
\item%
\begin{description}%
\item[]%
\textit{Haliaeetus leucogaster (LC)}%
\item[]%
\textbf{White Bellied Sea Eagle/White Breasted Sea Eagle}%
\end{description}%
\begin{description}%
\item[H: ]%
Kind of uncommon breeding resident in lowlands and up to lower hills, more common in dry lowlands and regular visitor to higher hills. Mainly could be observed in near viscinity of coasts,large tanks and also along rivers.%
\item[D: ]%
Primarily fish, scavenged carrion, and occasionally reptiles and crustaceans. They hunt by soaring and diving, snatching prey from the water's surface.%
\item[R: ]%
Around the university ground and the boart yard area%
\end{description}%
\end{enumerate}%
\item%
Alcedinidae%
\begin{enumerate}%
\item%
\begin{description}%
\item[]%
\textit{Pelargopsis capensis (LC)}%
\item[]%
\textbf{Stork Billed Kingfisher}%
\end{description}%
\begin{description}%
\item[H: ]%
Uncommon breeding resident from lowlands up to lower hills. Wooded banks of rivers and streams, lakes and paddyfields adjoining wooded areas,tanks,lagoons and mangrove edged creeks are the habitat where mostly can be seen. %
\item[D: ]%
Mainly consists of fresh water fish speices, crustatians, frogs and sometimes small rodents.%
\item[R: ]%
Boart yard and the surrounding areas of Bolgoda lake%
\end{description}%
\item%
\begin{description}%
\item[]%
\textit{Halcyon smyrnensis (LC)}%
\item[]%
\textbf{White Brested Kingfisher/White Throated Kingfisher}%
\end{description}%
\begin{description}%
\item[H: ]%
Common breeding resident that is found throughout the Island. Can be observed in open areas,cultivation,gardens and wetlands. Argubly the most common kingfisher species in Sri Lanka.%
\item[D: ]%
This predator primarily targets a diverse range of prey, including large crustaceans, insects, earthworms, rodents, lizards, snakes, fish, and frogs. There have been reported instances of predation on small birds, such as the Indian white{-}eye, chicks of red{-}wattled lapwings, sparrows, and munias. Notably, the young of this species are predominantly fed on invertebrates.%
\item[R: ]%
Boart yard and the surrounding areas of Bolgoda lake%
\end{description}%
\item%
\begin{description}%
\item[]%
\textit{Ceryle rudis (LC)}%
\item[]%
\textbf{Lesser Pied Kingfisher}%
\end{description}%
\begin{description}%
\item[H: ]%
Farily uncommon breeding resident in lowlands. ocassionally visits low hills. Paddyfields,streams,tanks and coastal areas of estuaries,lagoons,large tanks and slow moving rivers are the preffered habitat.%
\item[D: ]%
The primary diet of this species comprises fish, although it occasionally consumes crustaceans and large aquatic insects such as dragonfly larvae. Its hunting strategy typically involves hovering over the water to detect prey, followed by a rapid vertical dive with the bill leading, aimed at capturing fish.%
\item[R: ]%
Boart yard and the surrounding areas of Bolgoda lake%
\end{description}%
\item%
\begin{description}%
\item[]%
\textit{Alcedo atthis (LC)}%
\item[]%
\textbf{Common Kingfisher/Eurasian Kingfisher/ River Kingfisher}%
\end{description}%
\begin{description}%
\item[H: ]%
Somewhat uncommon breeding resident almost thoughout Sri Lanka, but rare in higher hills. Wetlands, open country and forests are the preffered habitat.%
\item[D: ]%
The common kingfisher's primary diet consists mainly of small fish, although it also includes insect larvae and, occasionally, frogs. When perched on a branch or reed above the water, the small bird patiently awaits the sight of a potential prey {-} a small fish in the water. Upon spotting its target, the kingfisher executes a swift, vertical plunge into the water with its wings retracted. In an attempt to catch the fish, the bird uses its beak before swiftly taking off again. To subdue and prepare its catch, the kingfisher often strikes the fish several times against a branch.%
\item[R: ]%
Boart yard and the surrounding areas of Bolgoda lake%
\end{description}%
\end{enumerate}%
\item%
Anatidae%
\begin{enumerate}%
\item%
\begin{description}%
\item[]%
\textit{Dendrocygna javanica (LC)}%
\item[]%
\textbf{Lesser Whistling Duck/Indian Whistling Duck/ Lesser Whistling Teal}%
\end{description}%
\begin{description}%
\item[H: ]%
Common breeding resident throughout the lowlands. Migrant population also has been identified during the northern winter. Can be commonly observed in fresh water marshes and tanks.%
\item[D: ]%
Mainly on plants taken from the water as well as grains from cultivated rice apart from small fish, frogs and invertebrates such as molluscs and worms%
\item[R: ]%
Boart yard and the surrounding areas of Bolgoda lake%
\end{description}%
\end{enumerate}%
\item%
Anhingidae%
\begin{enumerate}%
\item%
\begin{description}%
\item[]%
\textit{Anhinga melanogaster (LC)}%
\item[]%
\textbf{Oriental Darter}%
\end{description}%
\begin{description}%
\item[H: ]%
Uncommon breeeding resident in dry lowlands. Very rarely observed in wet lowlands and high hills. Tanks, lakes, larger rivers and lagoons are the preffered habitat.%
\item[D: ]%
Mainly fish, particularly snakeheads and mudfish.%
\item[R: ]%
Boart yard and the surrounding areas of Bolgoda lake%
\end{description}%
\end{enumerate}%
\item%
Apodidae%
\begin{enumerate}%
\item%
\begin{description}%
\item[]%
\textit{Cypsiurus balasiensis (LC)}%
\item[]%
\textbf{Asian Palm Swift}%
\end{description}%
\begin{description}%
\item[H: ]%
Common breeding resident throughout Sri Lanka. Mainly observed around fan palms.%
\item[D: ]%
The Asian Palm{-}swift primarily sustains itself on an insect{-}based diet. Its main food sources include grasshoppers, moths, crickets, mantises, cicadas, beetles, dragonflies, flying termites, airborne spiders, flies, locusts, and flying ants. Exhibiting exceptional aerial foraging skills, this swift adeptly hunts insects while in flight, showcasing its proficiency as an airborne predator.%
\item[R: ]%
Surrounding woody areas of the university ground%
\end{description}%
\item%
\begin{description}%
\item[]%
\textit{Aerodramus unicolor (VU)}%
\item[]%
\textbf{Indian Swiftlet/Indian Edible{-}Nest Swiftlet}%
\end{description}%
\begin{description}%
\item[H: ]%
Can be found throughout Sri Lanka in common and a breeding resident too. Roost and breeds inside caves.%
\item[D: ]%
Insectivorous by nature, these swiftlets skillfully capture their prey in mid{-}flight. Their diet encompasses a variety of insects, including flies, sawflies, wasps, bees, cicadas, flying termites, flying ants, beetles, locusts, grasshoppers, airborne spiders, and butterflies. Renowned for their exceptional agility in flight, these swiftlets demonstrate a unique drinking behavior by skimming the water surface while in motion.%
\item[R: ]%
Surrounding woody areas of the university ground%
\end{description}%
\end{enumerate}%
\item%
Ardeidae%
\begin{enumerate}%
\item%
\begin{description}%
\item[]%
\textit{Ardea intermedia (LC)}%
\item[]%
\textbf{Intermediate Egret/ Median Egret/ Smaller Egret/ Medium Egret}%
\end{description}%
\begin{description}%
\item[H: ]%
an uncommon breeding resident found from lowlands to hills. Commonly could be observed in marshes, paddyfields and shallow vegetated edges of tanks and lagoons.%
\item[D: ]%
Fish, frogs, insects, and crustaceans. They forage in shallow water, often stalking their prey.%
\item[R: ]%
Boart yard and the surrounding areas of Bolgoda lake%
\end{description}%
\item%
\begin{description}%
\item[]%
\textit{Ardea purpurea (LC)}%
\item[]%
\textbf{Purple Heron}%
\end{description}%
\begin{description}%
\item[H: ]%
A common breeding resident in lowlands and assends up to lower hills, but rare to spot in hills.Can be spotted in vegetation of marshes, paddyfields and tanks. %
\item[D: ]%
Fish, frogs, insects, and small mammals.%
\item[R: ]%
Boart yard and the surrounding areas of Bolgoda lake%
\end{description}%
\item%
\begin{description}%
\item[]%
\textit{Ardeola grayii (LC)}%
\item[]%
\textbf{Indian Pond Heron/Paddybird}%
\end{description}%
\begin{description}%
\item[H: ]%
Can be found around the edges of marshes, coastal lagoons and freshwater wetlands. Common breeding residents that can be found all throughout Sri Lanka and numerous winter migrants appear to arrive all over the country. %
\item[D: ]%
Mainly consists of crustaceans, aquatic insects, fishes, tadpoles and sometimes leeches. Outside wetlands, these herons regularly feed on insects (including crickets, dragonflies and bees)%
\item[R: ]%
Around the edges of the Bolgoda lake and open grassy areas throughout the university.%
\end{description}%
\item%
\begin{description}%
\item[]%
\textit{Nycticorax nycticorax (LC)}%
\item[]%
\textbf{Black Crowned Night Heron/Black Capped Night Heron}%
\end{description}%
\begin{description}%
\item[H: ]%
A Common breeding resident throughout the lowlands  up to lower hills. Occasionally assends to higher hills too. Can be observed in mrashes, mangroves and other vegetation at edge of rivers,tanks and lagoons as well as tree covered islands.%
\item[D: ]%
Fish, frogs, crabs, and insects. They are nocturnal hunters and forage in wetlands and rice paddies.%
\item[R: ]%
Boart yard and the surrounding marshy areas of the Bolgoda lake. %
\end{description}%
\item%
\begin{description}%
\item[]%
\textit{Ardea cinerea (LC)}%
\item[]%
\textbf{Grey Heron}%
\end{description}%
\begin{description}%
\item[H: ]%
Fairly common breeding resident in dry lowlands. Rare to see in wet lowlands. Occasional records are present at high hills up to 2000m. Coastal lagoons, marshes, estuaries and tanks are the preffered habitats.%
\item[D: ]%
Grey herons are carnivores, primarily piscivores, and their main diet consists of fish. However, their feeding habits can vary with the season and availability of prey. They may also consume amphibians, crustaceans, aquatic invertebrates, mollusks, snakes, small birds, rodents, and occasionally, certain plants.%
\item[R: ]%
Boart yard and the surrounding areas of Bolgoda lake%
\end{description}%
\end{enumerate}%
\item%
Charadriidae%
\begin{enumerate}%
\item%
\begin{description}%
\item[]%
\textit{Vanellus indicus (LC)}%
\item[]%
\textbf{Red Wattled Lapwing}%
\end{description}%
\begin{description}%
\item[H: ]%
Common breeding resident throughout the island but somewhat rare in hills. Can be commonly spotted in open flat grounds located in the near viscinity of water.%
\item[D: ]%
Insects, worms, and other invertebrates. They forage by running and probing the ground with their long bills.%
\item[R: ]%
Observed in the university ground premises and in the open area in front of Department of Civil Engineering.%
\end{description}%
\end{enumerate}%
\item%
Ciconiidae%
\begin{enumerate}%
\item%
\begin{description}%
\item[]%
\textit{Anastomus oscitans (LC)}%
\item[]%
\textbf{Asian Openbill/Asian Openbill Stork}%
\end{description}%
\begin{description}%
\item[H: ]%
Fairly common breeding resident in lowlands, can be found in small numbers along rivers in interior wet zone and on lakes and rivers up to lower hills. Marshes, tanks,lakes,rivers and lagoons are the preffered habitats.%
\item[D: ]%
The Asian Openbill primarily feeds on large mollusks, particularly Pila species, in shallow water and marshy areas. Using its distinct bill, characterized by a gap that forms with age, the bird separates snail shells from their bodies. The lower mandible's twisted tip aids in this process. The bill gap appears to be an adaptation for handling hard and slippery shells. Foraging involves slow walks with a steady gait, and the bird makes rapid vertical jabs to capture prey. The diet also includes smaller snails, water snakes, frogs, and large insects.%
\item[R: ]%
Boart yard and the surrounding areas of Bolgoda lake%
\end{description}%
\item%
\begin{description}%
\item[]%
\textit{Mycteria leucocephala (LC)}%
\item[]%
\textbf{Painted Stork}%
\end{description}%
\begin{description}%
\item[H: ]%
Farily common breeding resident in dry lowlands. An introduced population can be observed in and the suburbs of Colombo. Marshes,tanks and lagoons are the places to look for.%
\item[D: ]%
Painted storks are classified as carnivores, specifically piscivores. Their primary diet consists of small fish, and they also consume crustaceans, amphibians, insects, and reptiles. Additionally, they include frogs in their diet and occasionally prey on snakes.%
\item[R: ]%
Boart yard and the surrounding areas of Bolgoda lake%
\end{description}%
\end{enumerate}%
\item%
Cisticolidae%
\begin{enumerate}%
\item%
\begin{description}%
\item[]%
\textit{Orthotomus sutorius (LC)}%
\item[]%
\textbf{Common Tailorbird}%
\end{description}%
\begin{description}%
\item[H: ]%
Common breeding resident throughout Sri Lanka. Can be seen in forest wooded areas, adn trees in villages and town gardens.%
\item[D: ]%
These birds primarily subsist on insects, displaying a particular fondness for beetles and bugs. They are drawn to insects around flowers, with a notable preference for the inflorescences of mango trees. Additionally, they visit flowers to consume nectar and, at times, become covered in pollen, imparting a distinctive golden{-}headed appearance.%
\item[R: ]%
Mostly observed in the bushes and trees near library area in the side of Kaju kele. Also observed in the university ground at the side of lagan.%
\end{description}%
\end{enumerate}%
\item%
Columbidae%
\begin{enumerate}%
\item%
\begin{description}%
\item[]%
\textit{Columba livia (CR)}%
\item[]%
\textbf{Rock Pigeon/Rock Dove/ Common Pigeon}%
\end{description}%
\begin{description}%
\item[H: ]%
Local and rare breeding resident on offshre rocky islets from north{-}east to southern areas. Wild populations occur rarely at large concrete dams while the largest may be in the pigeon island. Feral populations are the most common living in villages and towns. %
\item[D: ]%
Mainly grains, seeds, plant seeds. They will also feast on berries, fruits and vegetables. Occasionally even will eat insects,snails and worms too.%
\item[R: ]%
Mostly observed in the library building. Uses broken ceilings as shelter there.%
\end{description}%
\item%
\begin{description}%
\item[]%
\textit{Ducula aenea (LC)}%
\item[]%
\textbf{Green Imperial Pigeon}%
\end{description}%
\begin{description}%
\item[H: ]%
Fairly common breeding resident. Found from lowlands to lower hills. Can be mostly seen in forests and well{-}wooded gardens.%
\item[D: ]%
Fruits, seeds, and leaves. They forage in trees and on the ground.%
\item[R: ]%
Surrounding areas of Lagan, Steel building, in Kaju kele and on the trees of Ceremonial courtyard.%
\end{description}%
\item%
\begin{description}%
\item[]%
\textit{Spilopelia suratensis (LC)}%
\item[]%
\textbf{Spotted Dove/Eastern Spotted Dove}%
\end{description}%
\begin{description}%
\item[H: ]%
Very common breeding resident found throughout the island except the high hills. Cultivation, gardens and the open forests are the preffered habitat and usually avoids interior of dense wet forests.%
\item[D: ]%
Seeds, fruits, and grain. They forage on the ground and in trees.%
\item[R: ]%
Almost throughout the university premises.%
\end{description}%
\end{enumerate}%
\item%
Corvidae%
\begin{enumerate}%
\item%
\begin{description}%
\item[]%
\textit{Corvus splendens (LC)}%
\item[]%
\textbf{House Crow/Indian/Grey Necked/Ceylon/Colombo Crow}%
\end{description}%
\begin{description}%
\item[H: ]%
Very common breeding resident in coastal areas. locally common in interior lowland areas up to lower hills and rare in occasional up to mid hills. Can be se in and near viscinity of human habitations.%
\item[D: ]%
These birds primarily feed on refuse in human habitations, targeting small reptiles and mammals. Additionally, their diet includes insects, other small invertebrates, eggs, nestlings, grain, and fruits. They exhibit a diverse and opportunistic feeding behavior.%
\item[R: ]%
Throughout the university. %
\end{description}%
\item%
\begin{description}%
\item[]%
\textit{Corvus macrorhynchos (LC)}%
\item[]%
\textbf{Jungle Crow/Large{-}Billed Crow}%
\end{description}%
\begin{description}%
\item[H: ]%
Farily common breeding resident throughout. Can be seen in villages and towns adjoining forests and forest patches.%
\item[D: ]%
This bird exhibits remarkable versatility in its feeding habits, as it forages on the ground or in trees. Its diet is extensive, encompassing a wide range of items, and it displays a tendency to attempt feeding on anything that appears edible, whether alive or dead, and from both plant and animal sources.%
\item[R: ]%
Throughout the university. %
\end{description}%
\end{enumerate}%
\item%
Cuculidae%
\begin{enumerate}%
\item%
\begin{description}%
\item[]%
\textit{Eudynamys scolopaceus (LC)}%
\item[]%
\textbf{Asian Koel/Western Koel}%
\end{description}%
\begin{description}%
\item[H: ]%
Common breeding resident from lowland to midhills. Forests, open woodlands gardens and cultivations are the habitats that mostly preffered.%
\item[D: ]%
As omnivores, these creatures have a diverse diet that includes insects, caterpillars and small vertebrates. However, adults primarily sustain themselves by consuming fruits%
\item[R: ]%
Can be seen throughout the university where there are trees.%
\end{description}%
\item%
\begin{description}%
\item[]%
\textit{Centropus sinensis (LC)}%
\item[]%
\textbf{Southern Coucal/Greater Coucal/ Common Coucal}%
\end{description}%
\begin{description}%
\item[H: ]%
Common breeding resident throughout the Island. Can be observed in forest edges of the wetland rain forests, dry forest, scrub and thickets in gardens and around cultivation.%
\item[D: ]%
Mainly  insects, caterpillars, snails, and small vertebrates. Also known to eat bird eggs, nestlings, fruits, and seeds.%
\item[R: ]%
In Kaju kele area and surroundings of Lagan.%
\end{description}%
\end{enumerate}%
\item%
Dicaeidae%
\begin{enumerate}%
\item%
\begin{description}%
\item[]%
\textit{Dicaeum erythrorhynchos (LC)}%
\item[]%
\textbf{Pale Billed Flowerpecker/Tickell'S Flowerpecker/ Small Flowerpecker}%
\end{description}%
\begin{description}%
\item[H: ]%
Farily commmon breeding resident throughout the Island. Preffers areas with tall trees.%
\item[D: ]%
 feeds on nectar and berries%
\item[R: ]%
Around the lagan, Seetha gangula and Ceremonial courtyard.%
\end{description}%
\end{enumerate}%
\item%
Dicruridae%
\begin{enumerate}%
\item%
\begin{description}%
\item[]%
\textit{Dicrurus caerulescens (LC)}%
\item[]%
\textbf{White Bellied Drongo}%
\end{description}%
\begin{description}%
\item[H: ]%
Common breeding resident from lowlands up to mid hills. Can be observed in forest fringe, wooded areas, plantations, gardens and towns.%
\item[D: ]%
While primarily insectivorous, these birds display opportunistic behavior and are known to prey on small birds. Similar to other drongos, they utilize their feet when handling prey. Additionally, they have been observed taking advantage of insects attracted to artificial lights during the late dusk period.%
\item[R: ]%
Can be seen throughout the university specially around Sumandasa building, Kaju Kele area and open area behind the Archi auditorium%
\end{description}%
\end{enumerate}%
\item%
Estrildidae%
\begin{enumerate}%
\item%
\begin{description}%
\item[]%
\textit{Lonchura punctulata (LC)}%
\item[]%
\textbf{Scaly Breasted Munia/Spotted Munia/ Spice Finch}%
\end{description}%
\begin{description}%
\item[H: ]%
Common breeding resident thoughout the Island. Cultivation and open gardens are the preffered habitat.%
\item[D: ]%
The primary diet of this bird consists mainly of grass seeds, small berries like those from the Lantana plant, and insects. Despite having a bill well{-}adapted for crushing small grains, they do not exhibit lateral movements of the lower mandible.%
\item[R: ]%
Trees between Sumanadasa building and the University ground. Bushy areas behind the Main building of Department of Civil Engineering, Open areas around around Lagan and Deparment of textile and apparel engineering.%
\end{description}%
\end{enumerate}%
\item%
Hirundinidae%
\begin{enumerate}%
\item%
\begin{description}%
\item[]%
\textit{Hirundo rustica (LC*)}%
\item[]%
\textbf{Barn Swallow}%
\end{description}%
\begin{description}%
\item[H: ]%
Three recognised subspecies. H.r. gutturalis is a very common winter migrant. H.r.rustica is a rare and uncommon migrant. H.r.tyleri is a very rare but very regular migrant. Often can be seen in open country and towns near water.%
\item[D: ]%
Consumes a diverse range of airborne insects, with a particular focus on flies, beetles, wasps, wild bees, winged ants, and true bugs. Additionally, includes moths, damselflies, grasshoppers, and various other insect species in its diet.%
\item[R: ]%
Can be seen in flight around the university ground in the season.%
\end{description}%
\item%
\begin{description}%
\item[]%
\textit{Cecropis hyperythra (LC)}%
\item[]%
\textbf{Ceylon Swallow}%
\end{description}%
\begin{description}%
\item[H: ]%
Uncommon breeding resident throughout Sri Lanka. Open areas at forest fringe, plantations,human habitation, grasslands and paddyfields are the preffered habitat of the ceylon swallow.%
\item[D: ]%
Mostly insectivorous, taking flying insects on the wing.%
\item[R: ]%
Can be seen in flight around the university ground.%
\end{description}%
\end{enumerate}%
\item%
Jacanidae%
\begin{enumerate}%
\item%
\begin{description}%
\item[]%
\textit{Hydrophasianus chirurgus (LC)}%
\item[]%
\textbf{Pheasant Tailed Jacana}%
\end{description}%
\begin{description}%
\item[H: ]%
Common breeding resident in the lowlands of Sri Lanka. Can be easily observed in marshes,lakes and tanks with floating vegetation.%
\item[D: ]%
The primary diet of the Pheasant{-}tailed Jacana revolves around insects found on the water surface and invertebrates gleaned from the roots and leaves of aquatic vegetation. Using its bill, the bird grasps roots and picks snails, crustaceans, and other invertebrates. Additionally, it extracts prey from the undersides of water{-}lily leaves. In a complementary dietary aspect, the Pheasant{-}tailed Jacana consumes small quantities of vegetation, including seeds and ovules of lotus and water{-}lilies.%
\item[R: ]%
Boart yard and the surrounding areas of Bolgoda lake%
\end{description}%
\end{enumerate}%
\item%
Laridae%
\begin{enumerate}%
\item%
\begin{description}%
\item[]%
\textit{Chlidonias hybrida (LC*)}%
\item[]%
\textbf{Whiskered Tern}%
\end{description}%
\begin{description}%
\item[H: ]%
Common winter migrant to lowlands but rare in the hills. Marshes, tanks, paddyfields, coastal lagoons,salt{-}pans, coastal waters, lakes, rivers and estuaries are the preffered habitats.%
\item[D: ]%
 Small fish, shrimp, and other marine invertebrates. They hunt by dipping their bills into the water while hovering or flying low over the surface.%
\item[R: ]%
Boart yard and the surrounding areas of Bolgoda lake%
\end{description}%
\end{enumerate}%
\item%
Leiothrichidae%
\begin{enumerate}%
\item%
\begin{description}%
\item[]%
\textit{Argya affinis (LC)}%
\item[]%
\textbf{Yellow Billed Babler}%
\end{description}%
\begin{description}%
\item[H: ]%
Common breeding resident from lowlands up to mid hills. uncommon and local further in higher hills. Mostly being on the ground can be seen in wooded areas and trees in villages and town gardens. Avoids forests in wet zone and adjoining hills. %
\item[D: ]%
As an omnivore, this species has a varied diet that includes insects, spiders, small fruits, grains, nectar, and occasionally, lizards or scavenged scraps of food from human habitations.%
\item[R: ]%
Lagan and surroundings and also in the area in front of the Auditorium building of Faculty of Architecture.%
\end{description}%
\end{enumerate}%
\item%
Megalaimidae%
\begin{enumerate}%
\item%
\begin{description}%
\item[]%
\textit{Psilopogon zeylanicus (LC)}%
\item[]%
\textbf{Brown Headed Barbet}%
\end{description}%
\begin{description}%
\item[H: ]%
Common breeding resident from lowlands up to mid hills. Uncommon but recoreded in higher hills. Forests,open woods.gardens and trees in cultivation are the habitat which can be observed mostly. %
\item[D: ]%
primarily sustains itself on a fruit{-}based diet, encompassing a range of wild fruits, figs, drupes, berries, as well as cultivated garden fruits, vegetables, and plantation fruits. Additionally, these barbets are recognized for gleaning insects, including ants, termites, cicadas, grasshoppers, dragonflies, mantids, crickets, locusts, centipedes, beetles, and moths, from the branches and trunks of trees.%
\item[R: ]%
Area behind the Steel building and In Kaju kele. %
\end{description}%
\item%
\begin{description}%
\item[]%
\textit{Psilopogon rubricapillus (LC)}%
\item[]%
\textbf{Ceylon Small Barbet/Crimson Fronted Barbet/ Sri Lanka Barbet}%
\end{description}%
\begin{description}%
\item[H: ]%
Endemic, fairly common locally from wet lowlands up to mid hills. Uncommon but recorded in dry zone. Forests and the open wooded country is the preffered habitat.%
\item[D: ]%
Mainly consists of fruits and insects.%
\item[R: ]%
Ceremonial Courtyard area%
\end{description}%
\end{enumerate}%
\item%
Meropidae%
\begin{enumerate}%
\item%
\begin{description}%
\item[]%
\textit{Merops philippinus (CR)}%
\item[]%
\textbf{Blue Tailed Bee Eater}%
\end{description}%
\begin{description}%
\item[H: ]%
Farily common winter migrants throughout the Island.. Small flocks do breed on eastern dry lowlands. Open areas and the forests are the preffered habitat.%
\item[D: ]%
primarily sustains itself by consuming flying insects, with a particular focus on bees, wasps, and hornets. It employs sorties from an open perch, skillfully capturing these insects mid{-}air as part of its feeding behavior.%
\item[R: ]%
University ground, On and around Sumanadasa Building, Can be seen in flight around the trees of  ENTC and Department of Architecture.%
\end{description}%
\end{enumerate}%
\item%
Monarchidae%
\begin{enumerate}%
\item%
\begin{description}%
\item[]%
\textit{Terpsiphone paradisi (LC)}%
\item[]%
\textbf{Asian Paradise Flycatcher}%
\end{description}%
\begin{description}%
\item[H: ]%
Two recognised subspecies. One being(T.p.ceylonensis) a farily common breeding resident found in dry lowlands and adjoining dy lower hills. The other(T.p.paradisi) is a farily common winter migrant througout. Open forests,groves and town gardens are the habitat where mostly can be spotted.%
\item[D: ]%
feeds on insects.  They usually hunt in the understory of densely canopied trees.%
\item[R: ]%
Observed in Kaju kele area and the tree tops near Department of Civil Engineering.%
\end{description}%
\end{enumerate}%
\item%
Motacillidae%
\begin{enumerate}%
\item%
\begin{description}%
\item[]%
\textit{Motacilla tschutschensis/Motacilla tschutschensis (LC*)}%
\item[]%
\textbf{Eastern/Western Yellow Wagtail}%
\end{description}%
\begin{description}%
\item[H: ]%
Common winter migrant to Sri Lanka. Can be observed in damp grasslands and marshes.%
\item[D: ]%
It consumes a diverse range of insects, such as midges, flies, beetles, aphids, ants, and various others. Additionally, its diet includes spiders, a small quantity of snails, worms, berries, and seeds.%
\item[R: ]%
Around the university ground premises.%
\end{description}%
\item%
\begin{description}%
\item[]%
\textit{Anthus rufulus (LC)}%
\item[]%
\textbf{Paddyfield Pipit}%
\end{description}%
\begin{description}%
\item[H: ]%
Farily common breeding resident throughout the country. Grasslands and low scrub are the preffered habitat.%
\item[D: ]%
Its primary diet consists of small insects, yet it also indulges in larger prey such as beetles, small snails, and worms while traversing the ground. Additionally, this bird is known to chase insects like mosquitoes or termites while in flight.%
\item[R: ]%
Observed in the ground premises in near viscinity of Lagan.%
\end{description}%
\end{enumerate}%
\item%
Muscicapidae%
\begin{enumerate}%
\item%
\begin{description}%
\item[]%
\textit{Copsychus saularis (LC)}%
\item[]%
\textbf{Oriental Magpie Robin}%
\end{description}%
\begin{description}%
\item[H: ]%
Common breeding resident throughout Sri Lanka. Gardens,scrub,open forest and cultivation are the preffered habitats while avoiding interior of wet forests.%
\item[D: ]%
As carnivores, specifically insectivores, their primary diet comprises insects and other invertebrates. On occasion, they may also consume flower nectar, geckos, leeches, centipedes, and even fish.%
\item[R: ]%
Around the university ground premises.%
\end{description}%
\item%
\begin{description}%
\item[]%
\textit{Muscicapa muttui (LC*)}%
\item[]%
\textbf{Brown Breasted Flycatcher/Layard'S Flycatcher}%
\end{description}%
\begin{description}%
\item[H: ]%
Uncommon winter migrant to wet lowlands and up to lower hills. local and less common in mid hills and dry lowlands. Forest, well wooded areas often near streams are the places to spot easily.%
\item[D: ]%
Being insectivorous, brown{-}breasted flycatchers primarily feed on insects. Their common food items include beetles, caterpillars, flies, and spiders.%
\item[R: ]%
Observed around the Jack trees located near the library%
\end{description}%
\item%
\begin{description}%
\item[]%
\textit{Saxicoloides fulicatus (LC)}%
\item[]%
\textbf{Indian Robin}%
\end{description}%
\begin{description}%
\item[H: ]%
Common breeding resident in dry lowlands,rare and local in wet lowlands up to midhills. Forest, open scrub, cultivation edges and the gardens are the habitats that mostly can be observed.%
\item[D: ]%
Their primary diet consists mainly of insects, although they are also known to consume frogs and lizards, particularly when feeding their young at the nest. These birds may engage in late evening foraging, targeting insects drawn to lights during this time.%
\item[R: ]%
Observed only once in the courtyard of the chemical engineering department.%
\end{description}%
\end{enumerate}%
\item%
Nectariniidae%
\begin{enumerate}%
\item%
\begin{description}%
\item[]%
\textit{Leptocoma zeylonica (LC)}%
\item[]%
\textbf{Purple Rumped Sunbird}%
\end{description}%
\begin{description}%
\item[H: ]%
Common breeding resident in lowlands and up to mid hills. Rare in areas above that. Cultivation,gardens and forest are the places for look for. %
\item[D: ]%
These birds primarily sustain themselves by feeding on nectar, occasionally supplementing their diet with insects, especially when tending to their young. While they can hover for brief periods, their typical behavior involves perching to extract nectar from flowers.%
\item[R: ]%
Many areas where there are flowers present including the surrounding area of Department of FD \& PD. Also observed in the trees around the Bhavana and Trees of Ceremonial courtyard.%
\end{description}%
\item%
\begin{description}%
\item[]%
\textit{Cinnyris lotenius (LC)}%
\item[]%
\textbf{Loten'S Sunbird/ Long{-}Billed Sunbird/ Maroon{-}Breasted Sunbird}%
\end{description}%
\begin{description}%
\item[H: ]%
Farily common breeding resident throughout Sri lanka. But less common in the high hills. Can be observed in forests, wooded areas and gardens.%
\item[D: ]%
During their quest for nectar, they frequently hover around flowers. Similar to other sunbirds, they also consume small insects and spiders. Their drinking habits encompass both garden plants and more untamed shrubs with equal enthusiasm.%
\item[R: ]%
Many areas where there are flowers present including the surrounding area of Department of FD \& PD. Also observed in the trees around the Bhavana and Trees of Ceremonial courtyard.%
\end{description}%
\end{enumerate}%
\item%
Oriolidae%
\begin{enumerate}%
\item%
\begin{description}%
\item[]%
\textit{Oriolus kundoo (LC*)}%
\item[]%
\textbf{Indian Golden Oriole}%
\end{description}%
\begin{description}%
\item[H: ]%
Rare winter migrant to lowlands and lower hills. Solitary and can be seen in wooded areas.%
\item[D: ]%
The diet of these golden oriole species consists mainly of wild fruits. %
\item[R: ]%
Oserved only once and it was in the greenery behind the Library%
\end{description}%
\item%
\begin{description}%
\item[]%
\textit{Oriolus xanthornus (LC)}%
\item[]%
\textbf{Black Headed Oriole/Black Hooded Oriole}%
\end{description}%
\begin{description}%
\item[H: ]%
fairly common breeding resident found in lowlands up to mid hills. Forests,wooded areas and trees in villages and town gardens are the habitats where can be easily seen.%
\item[D: ]%
The Black{-}Headed Oriole's adaptable diet, ranging from insects like caterpillars and beetles to fruits and nectar, positions them as important contributors to their ecosystem. Their diverse foraging habits support their roles as predators, pollinators, and seed dispersers, contributing to the overall health and balance of their habitats.%
\item[R: ]%
Trees around the Library, Kaju kele and around the trees of Building of Faculty of Architecture.%
\end{description}%
\end{enumerate}%
\item%
Pelecanidae%
\begin{enumerate}%
\item%
\begin{description}%
\item[]%
\textit{Pelecanus philippensis (LC)}%
\item[]%
\textbf{Spot Billed Pelican}%
\end{description}%
\begin{description}%
\item[H: ]%
Common breeding resident throughout the dry lowlands.Introduced population commonly can be observed in the wet zone.  Can be commonly observed in tanks,lakes,lagoons and marshlands.%
\item[D: ]%
Primarily fish, especially smaller to mid sized species including inrtoduced species like catfish, carp, and tilapia. They also occasionally consume frogs, crustaceans, and even small birds.%
\item[R: ]%
Boart yard and the surrounding areas of Bolgoda lake%
\end{description}%
\end{enumerate}%
\item%
Phalacrocoracidae%
\begin{enumerate}%
\item%
\begin{description}%
\item[]%
\textit{Phalacrocorax fuscicollis (LC)}%
\item[]%
\textbf{Indian Shag/Indian Cormorant}%
\end{description}%
\begin{description}%
\item[H: ]%
Very common breeding resident in dry lowlands. Less commonly observed in wet lowlands and lower hills too. Marshlands, tanks, lakes, large channels,rivers and lagoons are the preffered habitat.%
\item[D: ]%
Primarily fish, but also consumes crustaceans, mollusks, and aquatic insects.%
\item[R: ]%
Boart yard and the surrounding areas of Bolgoda lake%
\end{description}%
\item%
\begin{description}%
\item[]%
\textit{Phalacrocorax carbo (NT)}%
\item[]%
\textbf{Great Cormorant}%
\end{description}%
\begin{description}%
\item[H: ]%
Somehwat rare breeding resident in dry lowlands and very rarely seen in wet lowlands. Could be spotted in larger tanks and coastal lagoons.%
\item[D: ]%
Primarily fish, especially eel species and catfish. They also hunt frogs, crabs, and other aquatic animals.%
\item[R: ]%
Boart yard and the surrounding areas of Bolgoda lake%
\end{description}%
\item%
\begin{description}%
\item[]%
\textit{Microcarbo niger (LC)}%
\item[]%
\textbf{Little Cormorant}%
\end{description}%
\begin{description}%
\item[H: ]%
Very common breeding resident mainly from lowlands to lower hills. Present in higher hills but less common. Lakes, tanks, marshes, paddyfields,rivers,streams and lagoons are the places where mostly can be seen.%
\item[D: ]%
Little cormorant birds primarily consume fish, occasionally incorporating crustaceans and amphibians into their diet. They engage in diving to capture their prey and resurface to swallow it.%
\item[R: ]%
Boart yard and the surrounding areas of Bolgoda lake%
\end{description}%
\end{enumerate}%
\item%
Picidae%
\begin{enumerate}%
\item%
\begin{description}%
\item[]%
\textit{Dinopium psarodes (LC)}%
\item[]%
\textbf{Lesser Sri Lanka Flameback/Red{-}Backed Flameback/Sri Lanka Red{-}Backed Woodpecker/ Black{-}Rumped Flameback}%
\end{description}%
\begin{description}%
\item[H: ]%
Fairly common breeding resident. The southern red{-}backed race (psarodes) is being the recorded subspecies in the university premises. Forests,coconut plantations and trees in open country are the areas where can be easily observed.%
\item[D: ]%
The primary dietary focus of this bird centers on ants, particularly the pupae and larvae of Asian weaver ants. Additionally, it consumes various other invertebrates such as spiders, caterpillars, weevils, and beetles. While its main diet is insect{-}based, it also occasionally includes fruit in its food intake, serving as a source of dietary fiber and other essential nutrients.%
\item[R: ]%
Woody areas around the department of Architecture and Lagan.%
\end{description}%
\end{enumerate}%
\item%
Psittacidae%
\begin{enumerate}%
\item%
\begin{description}%
\item[]%
\textit{Loriculus beryllinus (LC)}%
\item[]%
\textbf{Ceylon Hanging Parrot}%
\end{description}%
\begin{description}%
\item[H: ]%
Endemic and farily commonly found in wet lowlands to mid hills. Restricted to the foothill areas in the dry lowlands. Can be mostly seen in forests and wooded areas.%
\item[D: ]%
Their natural diet consists mainly of fruits; particularly wild figs, guava and berries, as well as flower buds and blossoms. Also feeds onnectar and seeds.%
\item[R: ]%
Surrounding areas of university ground and Ceremonial courtyard.%
\end{description}%
\item%
\begin{description}%
\item[]%
\textit{Psittacula krameri (LC)}%
\item[]%
\textbf{Rose Ringed Parakeet/Indian Ringneck/ Kramer Parrot}%
\end{description}%
\begin{description}%
\item[H: ]%
Common breeding resident from lowlands to midhills and a small population recorded in higer hills. Forests and wooded areas by villages and towns are the preffered habittat. Can be observed even in densely populated areas aswell.%
\item[D: ]%
usually feed on buds, fruits, vegetables, nuts, berries, and seeds.%
\item[R: ]%
Can be seen in flight at evenings in the university ground premises.%
\end{description}%
\end{enumerate}%
\item%
Pycnonotidae%
\begin{enumerate}%
\item%
\begin{description}%
\item[]%
\textit{Pycnonotus cafer (LC)}%
\item[]%
\textbf{Red Vented Bulbul}%
\end{description}%
\begin{description}%
\item[H: ]%
Very common breeding resident that can be seen all around the country. Easy to spot in forest fringe, open forest, cultivated lands and trees in habitation.  Avoid interior of thick forests in wet zone.%
\item[D: ]%
Their diet encompasses a variety of food sources, including fruits, flower petals, nectar, and insects. Additionally, they occasionally consume house geckos.%
\item[R: ]%
Throughout the university where there are greenery.%
\end{description}%
\item%
\begin{description}%
\item[]%
\textit{Pycnonotus luteolus (LC)}%
\item[]%
\textbf{White Browed Bulbul}%
\end{description}%
\begin{description}%
\item[H: ]%
Common breeding resident mainly in dry lowlands. Uncommon in wet lowlands and up to midhills. Dry forests, scrub and gardens are the preffered habitat.%
\item[D: ]%
They search for food within bushes, where they primarily feed on a diet consisting of fruit, nectar, and insects.%
\item[R: ]%
In kaju Kele and around the trees near Archi Main Building%
\end{description}%
\end{enumerate}%
\item%
Rallidae%
\begin{enumerate}%
\item%
\begin{description}%
\item[]%
\textit{Amaurornis phoenicurus (LC)}%
\item[]%
\textbf{White Breasted Waterhen}%
\end{description}%
\begin{description}%
\item[H: ]%
Common breeding resident throughout the Island. Marshes, paddy fields and any other wet areas with thick cover are the preffered habitat.%
\item[D: ]%
Aquatic plants, seeds, fruits, insects, and small fish. They forage by wading in shallow water and picking food from the surface.%
\item[R: ]%
Boart yard and the surrounding shallow areas of Bolgoda lake%
\end{description}%
\item%
\begin{description}%
\item[]%
\textit{Porphyrio poliocephalus (LC)}%
\item[]%
\textbf{Purple Swamphen}%
\end{description}%
\begin{description}%
\item[H: ]%
Locally common breeding resident in the lowlands. Reedbeds, marshes, paddyfields and weedy tanks are the habitat types which can be easily spotted.%
\item[D: ]%
 Omnivorous, feeding on insects, worms, frogs, small fish, seeds, fruits, and leaves.%
\item[R: ]%
Boart yard and the surrounding shallow areas of Bolgoda lake%
\end{description}%
\end{enumerate}%
\item%
Strigidae%
\begin{enumerate}%
\item%
\begin{description}%
\item[]%
\textit{Ninox scutulata (NT)}%
\item[]%
\textbf{Brown Hawk Owl/Brown Boobook}%
\end{description}%
\begin{description}%
\item[H: ]%
A common breeding resident. Can be found throughout the Island except in the higher hills. Can be observed in forests and other well wooded areas of villages and towns. %
\item[D: ]%
The primary dietary focus of the brown hawk{-}owl revolves around insects, encompassing large insects as well as small mammals, birds, frogs, and lizards.%
\item[R: ]%
Can be observed around Lagan at night.%
\end{description}%
\item%
\begin{description}%
\item[]%
\textit{Otus bakkamoena (LC)}%
\item[]%
\textbf{Indian Scops Owl/Collared Scops Owl}%
\end{description}%
\begin{description}%
\item[H: ]%
Fairly common breeding resident found throughout the entire Island. uncommon in the higher hills. avoid interior of wet evergreen forests and typically found inside well wooded areas and residantial gardens of villages and towns.%
\item[D: ]%
The primary diet of the Indian Scops{-}owl primarily comprises insects, including beetles and grasshoppers, along with various other species. Additionally, it incorporates moth larvae into its diet. On occasion, this owl may also prey on small vertebrates such as rodents, small birds, and lizards, with bats being a rare and infrequent inclusion in its diet.%
\item[R: ]%
Can be observed in seetha gangula area at night%
\end{description}%
\end{enumerate}%
\item%
Sturnidae%
\begin{enumerate}%
\item%
\begin{description}%
\item[]%
\textit{Acridotheres tristis (LC)}%
\item[]%
\textbf{Common Myna/Indian Myna}%
\end{description}%
\begin{description}%
\item[H: ]%
Common breeding throughout. Forest edges, trees in open areas, village and town gardens are the places to look for. %
\item[D: ]%
Being omnivorous, this species has a diverse diet that includes insects, arachnids, crustaceans, reptiles, small mammals, seeds, grains, fruits, and discarded waste from human habitation.%
\item[R: ]%
Can be seen throughout the university including inside the goda and goda uda canteen buildings.%
\end{description}%
\end{enumerate}%
\item%
Threskiornithidae%
\begin{enumerate}%
\item%
\begin{description}%
\item[]%
\textit{Threskiornis melanocephalus (LC)}%
\item[]%
\textbf{Black Headed Ibis/Oriental White Ibis/ Indian White Ibis/ Black{-}Necked Ibis}%
\end{description}%
\begin{description}%
\item[H: ]%
Common breeding resident in lowlands.Regularly observed near freshwater tanks, marshes and paddyfields.%
\item[D: ]%
Insects, worms, snails, and small frogs. They forage in wetlands, rice paddies, and fields, probing the ground with their long bills.%
\item[R: ]%
Boart yard and the surrounding areas of Bolgoda lake%
\end{description}%
\end{enumerate}%
\end{itemize}